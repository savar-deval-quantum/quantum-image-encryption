\documentclass[svgnames]{article}     % use "amsart" instead of "article" for AMSLaTeX format
%\geometry{landscape}                 % Activate for rotated page geometry

%\usepackage[parfill]{parskip}        % Activate to begin paragraphs with an empty line rather than an indent

\usepackage{graphicx}                 % Use pdf, png, jpg, or eps§ with pdflatex; use eps in DVI mode

%maths                                % TeX will automatically convert eps --> pdf in pdflatex
\usepackage{amssymb}
\usepackage{amsmath}
\usepackage{esint}
\usepackage{geometry}
\usepackage{hyperref}

\hypersetup{
  colorlinks=true, 
  linkcolor=magenta,
}

% Inverting Color of PDF
%\usepackage{xcolor}
%\pagecolor[rgb]{0.19,0.19,0.19}
%\color[rgb]{0.77,0.77,0.77}

%noindent
\setlength\parindent{0pt}

%pgfplots
\usepackage{pgfplots}

%images
\graphicspath{{/Users/devaldeliwala/screenshots/}}                   % Activate to set a image directory

%tikz
\usepackage{pgfplots}
\pgfplotsset{compat=1.15}
\usepackage{comment}
\usetikzlibrary{arrows}
\usepackage[most]{tcolorbox}

%Figures
\usepackage{float}
\usepackage{caption}
\usepackage{lipsum}


\title{Past Research}
\author{Deval Deliwala}
%\date{}                              % Activate to display a given date or no date

\begin{document}
\maketitle
%\section{}
%\subsection{}
%\tableofcontents                     % Activate to display a table of contents


\section{NEQRX: Efficient Quantum Image Encryption with Reduced Circuit
Complexity} 

\href{https://arxiv.org/pdf/2204.07996}{Reference} 

\vspace{10px}

Various methods have emerged to compute and store digital image information on
a quantum computer. The main two methods are 

\subsection{Flexible Representation of Quantum Images (FQRI)} 

Using FQRI, the positions $(x, y)$ of every pixel in an $n\times n$ image is encoded in a quantum
register containining $2n$ qubits. The intensity of each pixel is encoded using
the rotational angle of a separate qubit. \\

Therefore each of the $2n$ qubits contain their \textit{own} qubit
$|\psi\rangle$, where 

\[
|\psi\rangle = \cos\theta |0\rangle + \sin\theta | 1 \rangle.
\] \vspace{3px}

and $\theta$ is the angle representing the pixel's intensity. Therefore, an
image with $N$ pixels (where $N = 2^n$) is represented as a quantum state: 

\[
|\psi\rangle = \frac{1}{N}\sum_{Y=0}^{N-1} \sum_{X=0}^{N-1} \left(
  \cos\theta_{YX} 
| 0 \rangle + \sin\theta_{YX} | 1 \rangle\right) \otimes |YX\rangle
\] \vspace{3px}

Here $|YX\rangle$ represents the binary encoding of the pixel position, and
$\theta_i$ represents the intensity angle for pixel at position $(X, Y)$. 


\paragraph{Example} \mbox{} \\

For a 2x2 image with grayscale values, the encoding process can be summarized
as follows: 

\begin{itemize}
  \item[1.] \textbf{Image Matrix}: Suppose the image matrix is
\end{itemize}
\[
\begin{bmatrix}
  I_{00} & I_{01} \\ I_{10} & I_{11}
\end{bmatrix}
\] \vspace{3px}

where $I_{ij}$ represents the intensity of the pixel at position  $(i, j)$. 

 \begin{itemize}
   \item[2.] Encode the positions using 2 qubits (for 2x2 image, 4 positions:
     00, 01, 10, 11). 
    \item[3.] Encode the intensity values using rotation angles: $\theta_{00},
      \theta_{01}, \theta_{10}, \theta_{11}$. 
\end{itemize}

The quantum state representing the image $I$ will then be 

\[
|I\rangle = \frac{1}{2}\left( \cos\theta_{00}| 0 \rangle
+ \sin\theta_{00}| 1 \rangle \right) |00\rangle + \frac{1}{2}\left(
\cos\theta_{01}| 0 \rangle + \sin\theta_{01} | 1 \rangle \right)|01\rangle
+ \cdots
\] \vspace{3px}



Afterwards a bunch of quantum gates are applied to confuse and
diffuse the quantum states. Then measurement gates are applied collapsing it to
a classical state, from which the encoded pixel values and positions can be
retrieved. \\

The time complexity of quantum image preparation for FRQI is too high. For
a $2^n \times 2^n$ image, the procedure costs  $O(2^{4n})$, quadratic in the
image size. In addition because the gray-scale information of the image pixels
is stored as the probability amplitude of a single qubit, accurate image
retrieval is impossible for FRQI. 

\subsection{NEQR's Improvement} 

\href{https://zero.sci-hub.se/2180/25c5b2e7104ba68813ee5fc81176fb62/zhang2013.pdf}{Reference}\\

\textit{Novel Enhanced Quantum Representation} (NEQR) is very similar to FRQI,
except for its approach to encoding pixel values. \\

FRQI uses only a single qubit to store the gray-scale information for each
pixel. NEQR uses two-entangled qubit sequences to store the gray-scale and
position information, and stores the whole image in the superposition of the
two qubit sequences. \\

Suppose the gray range of an image is $2^q$, then a binary sequence  $C_{YX}^0
C_{YX}^1 \hdots C_{YX}^{q-2} C_{YX}^{q-1}$ encodes the gray scale value  $f(Y,
X)$ of the corresponding pixel $(X, Y)$. Mathematically, 

\[
  f(Y, X) = C_{YX}^0 C_{YX}^1 \hdots C_{YX}^{q-2} C_{YX}^{q-1}, \; C_{YX}^k \in
  [0, 1], \qquad f(Y, X) \in [0, 2^{q-1}]
\] \vspace{3px}

Then the representative expression of a quantum image for a $2^n \times 2^n$
image can be written as 

 \[
|I\rangle = \frac{1}{2^n} \sum_{Y=0}^{2^n - 1} \sum_{X=0}^{2^n - 1} |f(Y,
X)\rangle |YX\rangle = \frac{1}{2^n} \sum_{Y=0}^{2^n - 1} \sum_{X=0}^{2^n-1}
\otimes_{i=1}^{q-1} |C_{YX}^i\rangle |YX\rangle     
\] \vspace{3px}
\begin{figure}[H]
  \centering
    \includegraphics[width = 12cm]{screenshot 12.png}
    \caption{a $2\times 2$ example image and its representative expression in
    NEQR}
\end{figure}

The paper in the second reference provides extensive detail on how to
accomplish encoding a digital image in NEQR. \\

After quantum image preparation has been finished, you can manipulate
everything reversibly or apply chaotic maps, logistic maps, to confuse and
diffuse the image in a reversible manner. \\

The specific paper that implements \textit{NEQRX}, just preforms NEQR but
afterwards adds a generalized affine transformation and a logistic map. They
also use an \textit{Espresso} algorithm that apparently reduces computational
cost by 50\%. 

Almost \textit{every} quantum image encryption algorithm can be divided into
two main types: 

\begin{itemize}
  \item[1.] Transforming the image into a frequency domain with random
    operations
  \item[2.] Encrypting the image using chaos theory
\end{itemize}

This paper goes into detail about the statistical tests you can employ to
determine the effectiveness of an encryption algorithm -- mainly
\textit{Differential Analysis} and a \textit{Histogram Analysis}. It also
performs a \textit{Complexity Analysis} to assess the complexity of a quantum
circuit. \\

It afterwards performs a \textit{Noise Analysis}, where they use 6 known noisy
backends and analyze the effectiveness of their algorithm and its dependence on
noise.

\section{Quantum Image Encryption Based on Quantum DNA Codec and Pixel-Level
Scrambling}

This encryption algorithm creates a quantum DNA codec that encodes and decodes
the pixel color information of a quantum image using ``its special biological
properties". It afterwards employs \textit{quantum Hilbert Scrambling} to
muddle the position data in order to double the encryption effect. Then, the
altered image was then employed as a key matrix in a quantum XOR operation with
the original image. \\

\subsection{NCQI Model} \mbox{} \\

They use the NCQI model for encoding pixel information in quantum states. It is
fairly similar to FIQR, where the NCQI model of a $2^n \times 2^n$ image
$|I\rangle$ can be expressed as 

\[
  |I\rangle = \frac{1}{2^n} \sum_{Y=0}^{2^n - 1} \sum_{X=0}^{2^n - 1}
  |C_{YX}\rangle \otimes |YX\rangle
\] \vspace{3px}

where $|C_{YX}\rangle$ represents the color value of the pixel, which again is
encoded by a binary sequence $R_{q-1}\hdots R_0 G_{q-1}\hdots G_0 B_{q-1}\hdots
B_0$. \\


\subsection{DNA Coding Method \& Operation}

The ATCG binding rules for DNA state that A \& T are complementary and C \&
G are complementary. Similarly, in binary, 0 and 1 are complementary, and they
make certain binary sequences correspond to a DNA nucleoside. For example, for
a $2 \times 2$ image, with 8 coding schemes in accordance with the rules of
a biological model that encodes each nucleic acid with a 2-bit binary number, the DNA coding rules are as follows: 

\begin{figure}[H]
  \centering
    \includegraphics[width = 10cm]{screenshot 13.png}
    \caption{DNA coding rules}
\end{figure}


Scheme 1 produces CACT is an image pixel's R-channel gray value is 71, which is
represented by the binary sequence 01000111. 

\subsection{Quantum Hilbert Scrambling} 

Sounds fancy, but they just take the following matrix: 

\[
H_n = \begin{pmatrix}
  1 & 2 & 3 & \hdots & 2^n \\ 2^n + 1 & 2^n + 2 & 2^n + 3 & \hdots & 2^n + 1 \\
  \vdots & \vdots & \vdots & \vdots & \vdots \\
  2^{2n-1} + 1 & 2^{2^n -1} + 2 & 2^{2^n-1} + 3 & \hdots & 2^{2n}
\end{pmatrix} 
\] \vspace{3px}

which employs the following transformation,

\begin{figure}[H]
  \centering
    \includegraphics[width = 8cm]{screenshot 14.png}
    \caption{Results of performing a single Hilbert image scrambling}
\end{figure}

and then apply the following: 

\begin{figure}[H]
  \centering
    \includegraphics[width = 10cm]{screenshot 15.png}
\end{figure}

where $A^T$ represents a matrix $A$'s transpose,  $A^{ud}$ its upper and lower
direction reversed,  $A^{ld}$ its left and right inversion, and  $A^{pp}$ (haha
pp) its center rotation matrix.\\

They then talk about how they encoded the DNA codec, and applied the quantum
XOR operation, and afterwards go into the same statistical
analysis as the previous paper.  


\section{Conclusion}

Overall, these were the main two quantum image encryption algorithms that
encapsulate all previous versions of quantum image encryption. There are many
models like FIQR, NEQR, or NICQ that display a way to encode an image into
a quantum state, but all of them are roughly the same apart from different time
complexities to prepare the quantum state. \\

Our plan, which I think I have perfected, creates a completely new way to
encode pixel color information, though it requires more qubits (unless we
discretize an image into blocks of qubits, while somehow retaining the
information of all pixels in a block, maybe by entangling the block-statevector
with another qubit which holds all the RGB info for each pixel \textit{in} the
block?). 

I'll start working on the paper that goes into the plan and have it ready soon.

\subsection{References}

\subsubsection{Chaos Theory Papers}

\href{https://zero.sci-hub.se/2180/25c5b2e7104ba68813ee5fc81176fb62/zhang2013.pdf}{NEQR}\\
\href{https://www.researchgate.net/publication/235800177_A_flexible_representation_of_quantum_images_for_polynomial_preparation_image_compression_and_processing_operations_Quantum_Inf}{FIQR}\\
\href{https://arxiv.org/pdf/2204.07996}{NEQRX} \\
\href{https://www.frontiersin.org/journals/physics/articles/10.3389/fphy.2024.1230294/full}{Four-Dimensional
Chaos} \\
\href{https://www.ncbi.nlm.nih.gov/pmc/articles/PMC8871303/}{Logistic Quantum
Chaos}\\




There are many others, but I do not have access to them and they are all
roughly the same -- apply a logistic map, or rounds of different logistic maps
to scramble, then apply some color diffusing technique to gray-ify the image
and so on. The other types of image encryption, that just perform scrambling by
applying a bunch of other transformations are similar to the DNA codec paper or any other
classical technique, while still applying NEQR or FIQR. \\

The good news is that our idea is very novel and has a different way of encoding
information outside of FIQR or NEQR, and lowkey, is better than all of
them ong. 




\end{document}

